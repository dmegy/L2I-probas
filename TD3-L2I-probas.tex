\documentclass[11pt,a4paper]{article}

\usepackage[utf8]{inputenc} % ncessaire sur PC
\usepackage[T1]{fontenc}

\usepackage{mathtools,amssymb,amsthm}
\usepackage{ifthen}
\usepackage{verbatim}
\newboolean{enonce} 
%\setboolean{enonce}{false}
\setboolean{enonce}{true}
\newboolean{indication} 
%\setboolean{indication}{true}
\setboolean{indication}{false}
%\usepackage{eepic}

\usepackage[dvips]{graphicx}

\usepackage[english,frenchb]{babel}
%\usepackage{eurosym}
\usepackage{fancyhdr}

\newcounter{encompt}

\theoremstyle{definition}
\newtheorem{exo}{Exercice}

\newcommand{\refex}[1]{\thechapter.\ref{#1}}
\pagestyle{empty}
\newcommand{\ind}{ 1\hspace{-.55ex}\mbox{l}}
\setlength{\topmargin}{-0.5in}
\setlength{\textheight}{9in}
\setlength{\oddsidemargin}{0in}
\setlength{\evensidemargin}{0in}
\setlength{\textwidth}{6.5in}

\newcommand{\Card}{\mathrm{Card \,}}
\def\N{\mathbb N}
\def\R{\mathbb R}
\def\E{\mathbb E}
\renewcommand{\P}{\mathbb P}
\def\Sn{\frak{S}_n}

\newcommand{\titre}[1]{\pagestyle{fancy}\lhead{Probabilit\'es \hfill L2 Informatique -- S4} 
\rhead{}\begin{center}
\textbf{#1}
\end{center}\bigskip}


\begin{document}

\titre{Feuille 3 : variables aléatoires discrètes}


\begin{exo}
\begin{enumerate}
\item On lance deux dés équilibrés. On note $X_1$ le résultat du premier dé, et $X_2$ le résultat du deuxième dé. Déterminer la loi de $X_1-X_2$.
\item On lance un dé équilibré, au plus $5$ fois, en s'arrêtant dès qu'on obtient $6$. Donner la loi du nombre de lancers effectués.
\end{enumerate}
\end{exo}


\begin{exo}
Un livre de 350 pages contient 450 erreurs d'impression réparties au hasard. Soit $X$
la variable aléatoire du nombre d'erreurs dans une page déterminée.
\\
\textbf{a.} Quelle est la loi de $X$?
\\
\textbf{b.} Donner une expression de la probabilité qu'il y ait au moins 3 erreurs dans
      une page déterminée.
\end{exo}


\begin{exo}
Montrer que si $X$ est une v. a. de loi géométrique, elle vérifie 
la propriété d'absence de mémoire suivante: $\forall k\in\N,\;\forall n\in\N,\quad \P(X>n+k\mid X>n)=\P(X>k).$\\
Interpréter ce résultat en considérant une suite d'épreuves répétées.
\end{exo}


\begin{exo} Soient $X$ et $Y$ deux v. a. indépendantes de loi uniforme sur $\{0,1,2,\cdots,9\}$.

\noindent\textbf{a.} Calculer $\P(X=Y)$.

\noindent\textbf{b.} Déterminer la loi de $X+Y$.
%
%\noindent\textbf{c.} Soient $U=\min(X,Y)$ et $V=X-Y$. Calculer les lois de $U$ et $V$. %Calculer les lois de $(U,V)$, $U$ et $V$. 
%Les variables $U$ et $V$ sont-elles indépendantes ?
\end{exo}


\begin{exo} 
On lance deux dés. On note $X$ le plus grand des numéros obtenus, et $Y$
le plus petit.

\noindent
\textbf{a.} Déterminer les lois de $X$ et de $Y$. Ces deux variables aléatoires
sont-elles indépendantes~?\\
\noindent
\textbf{b.} Calculer $\E(X)$ et $\E(Y)$, et vérifier que $\E(X)+\E(Y)=7$. Comment aurait-on pu retrouver ce résultat de manière plus simple ?\\
\noindent
\textbf{c.} Calculer $V(X)$ et $V(Y)$.
\end{exo}


\begin{exo}
Soit $X$ une variable aléatoire discrète d'espérance et de variance finies et
$Y$ une autre v. a. telle que $Y = aX + b$. Calculer $\E({Y})$ et $V({Y})$ en fonction
de $\E(X)$ et $V(X)$.
\end{exo}


\begin{exo} \begin{enumerate}
\item Déterminer et tracer la fonction de répartition de la loi uniforme sur $\{0,\dots,n\}$ et de la loi géométrique de paramètre $p$.
\item Calculer l'espérance et la variance d'une variable aléatoire qui suit une loi uniforme sur l'ensemble $\{0,1,\ldots,n\}$.
\item Calculer l'espérance d'une variable aléatoire qui suit une loi géométrique de paramètre $p$.
%\item Calculer l'espérance d'une variable aléatoire qui suit une loi de Poisson de paramètre $\lambda$.
\end{enumerate}
\end{exo}


\begin{exo}Soient $X$ et $Y$ deux v. a. indépendantes, et de même loi :
$$\P(X=1)=\P(X=2)=\P(X=3)=\P(Y=1)=\P(Y=2)=\P(Y=3)=1/3.$$ 
On considère deux nouvelles v.a. $Z=X+Y$ et $T=X-Y$.

\begin{enumerate}
\item Donner les lois de $Z$ et $T$.
\item Montrer que les v.a. $Z$ et $T$ ne sont pas indépendantes.
\item Calculer $\E(Z),\E(T)$ et $\E(ZT)$ (on remarquera que $\E(ZT)=
\E(Z)\E(T)$ bien que $Z$ et $T$ ne soient pas indépendantes).
\end{enumerate}
\end{exo}


\begin{exo} 
Trois amis se retrouvent à une terrasse ensoleillée et commandent 3 cafés. Chacun d'eux met dans sa tasse 2 sucres avec probabilité ${1\over 6}$, $1$ sucre avec probabilité ${1\over 3}$, et pas de sucre avec probabilité ${1\over 2}$. Leurs choix sont bien entendu indépendants. On note $X_3$ le nombre de sucres consommés
par les 3 clients, et $Y$ le nombre de sucres consommés par le plus âgé.\\
\noindent
\textbf{a.} Donner la moyenne et la variance de $X_3$ et $Y$. \\ 
\noindent
\textbf{b.} Ces variables aléatoires sont-elles indépendantes ?
\end{exo}


\begin{exo}Soient $X,Y$ deux v. a. indépendantes prenant un nombre
fini de valeurs, respectivement $(x_i)_{i=1,\ldots,n}$ et $(y_j)_{j=1,\ldots,m}$. 
En considérant toutes les valeurs possibles pour le couple
$(X,Y)$, montrer que $\E(XY)=\E(X)\E(Y)$.
\end{exo}


\begin{exo} Une puce se déplace par sauts successifs sur les sommets A, B, C, et le centre de gravité O d'un triangle équilatéral. Au temps $t=0$, elle est en O. Puis elle saute en l'un des points A, B ou C de façon équiprobable. Par la suite, elle saute au temps $t=n$ du point où elle se trouve en l'un des autres points de façon équiprobable.
\\
\textbf{a.} Calculer la probabilité qu'elle revienne en O pour la première fois au temps $t=n$.
\\
\textbf{b.} Calculer la probabilité de l'événement : ``la puce revient en O''. Commenter.
\end{exo}


\begin{exo} Un examen consiste en 20 questions auxquelles il faut répondre par oui ou par non ; 
chaque réponse juste est notée 1 point et chaque réponse fausse, 0 point. 
Un étudiant répond entièrement au hasard : sa note finale est une variable aléatoire $X$.
\\
\textbf{a.} Soit $X_k$ la note qu'il obtient à la $k$--ième question : calculer 
$\E(X_k)$ et $V(X_k)$.
\\
\textbf{b.} Exprimer $X$ en fonction des $X_k$ ; en déduire $\E(X)$ et $V(X)$.
\\
\textbf{c.} Donner la loi de $X$ ; calculer la probabilité pour l'étudiant d'avoir une note inférieure à 3. 
\\
\textbf{d.} Un étudiant sérieux estime qu'il donnera une réponse exacte à chaque question avec une probabilité de 0,8. 
Quelle est la loi de sa note, son espérance et sa variance ? 
\end{exo}


\begin{exo} Soit $X = (X_1,X_2)$ une v.a. discrète à valeur dans $E=E_1\times E_2$, les $E_i$ étant finis ou dénombrables. La loi de $X$ est déterminée par la donnée, pour tout couple $(i,j)$ de $E$, de :
$$
\P({X_1=i \text{ et } X_2=j}) =: p_{i,j}.
$$
\noindent \textbf{a.} Calculer les lois de $X_1$ et $X_2$ en fonction de $p_{i,j}$.

\noindent \textbf{b.} On suppose que $E_1=E_2$, calculer $\P({X_1=X_2})$.

\noindent\textbf{c.} On suppose à partir de cette question que $E_i=\N$. 
Calculer la loi de $X_1+X_2$ en fonction de $p_{i,j}$.

\noindent\textbf{d.} \textit{Application}: pour $\lambda$ et $\mu$ positifs stricts, $X_1$ et $X_2$ indépendantes, et 
$$
\P({X_1=k}) = e^{-\lambda}\frac{\lambda^k}{k!}, \qquad 
\P({X_2=k}) = e^{-\mu}\frac{\mu^k}{k!}, \qquad k\geq 0. 
$$
Calculer $p_{i,j}$ puis donner une expression simple de la loi de $X_1+X_2$.
\end{exo}


\begin{exo} Dans une promotion de $n$ étudiants, chaque étudiant a une probabilité $p$ de réussir les épreuves écrites et d'être ainsi admis à passer l'oral. Chacun de ceux qui passe l'oral a une probabilité $a$ de le réussir. On note $X$ le nombre d'étudiants admis à passer l'oral, et $Y$ le nombre d'étudiants obtenant finalement le diplôme.

\noindent\textbf{a.} Quelle est la loi de $X$ ? Sachant que $X=k$, quelle est la loi de $Y$ ?

\noindent\textbf{b.} Donner la loi jointe $p_{k,l}$ de $(X,Y)$. En déduire la loi de $Y$.
\end{exo}


\begin{exo} Un assistant distrait range $n$ lettres au hasard  dans les enveloppes qu'il avait préalablement remplies. On note $N$ la variable aléatoire représentant le nombre de courriers qui arrivent effectivement à leur destinataire.
Calculer $\E(N)$.
%\textit{Question subsidiaire:} Calculer $V{N}$.  
%\\
%\textit{Question à $1\,000$ euros :} Calculer la loi de $N$. Calculer la loi limite de $N$.
\end{exo}

\end{document}