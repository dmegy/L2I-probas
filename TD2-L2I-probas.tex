\documentclass[11pt,a4paper]{article}

\usepackage[utf8]{inputenc} % ncessaire sur PC
\usepackage[T1]{fontenc}

\usepackage{mathtools,amssymb,amsthm}
\usepackage{ifthen}
\usepackage{verbatim}
\newboolean{enonce} 
%\setboolean{enonce}{false}
\setboolean{enonce}{true}
\newboolean{indication} 
%\setboolean{indication}{true}
\setboolean{indication}{false}
%\usepackage{eepic}

\usepackage[dvips]{graphicx}

\usepackage[english,frenchb]{babel}
%\usepackage{eurosym}
\usepackage{fancyhdr}

\newcounter{encompt}

\theoremstyle{definition}
\newtheorem{exo}{Exercice}

\newcommand{\refex}[1]{\thechapter.\ref{#1}}
\pagestyle{empty}
\newcommand{\ind}{ 1\hspace{-.55ex}\mbox{l}}
\setlength{\topmargin}{-0.5in}
\setlength{\textheight}{9in}
\setlength{\oddsidemargin}{0in}
\setlength{\evensidemargin}{0in}
\setlength{\textwidth}{6.5in}

\newcommand{\Card}{\mathrm{Card \,}}
\def\N{\mathbb N}
\def\R{\mathbb R}
\def\E{\mathbb E}
\renewcommand{\P}{\mathbb P}
\def\Sn{\frak{S}_n}

\newcommand{\titre}[1]{\pagestyle{fancy}\lhead{Probabilit\'es \hfill L2 Informatique -- S4} 
\rhead{}\begin{center}
\textbf{#1}
\end{center}\bigskip}


\begin{document}

\titre{Feuille 2 : Indépendance et probabilités conditionnelles}


\subsection*{Notion d'indépendance}

\begin{exo}
Soient $A,B,C$ des évènements. Pour chacune des assertions suivantes, donner soit une preuve,
soit un contre-exemple.
\begin{enumerate}
    \item $A$ est indépendant de $A$ si et seulement si $\mathbb{P}(A)=0$ ou $1$.
    \item $A$ est indépendant de $B$  si et seulement si $A$ est indépendant de $\overline{B}$.
    \item Si $A$ est indépendant de $B$ et $C$, alors il l'est de $B\cup C$.
    \item Si $A, B$ et $C$ sont indépendants, alors $A$ est indépendant de $B\cup C$.
    \item Si $A$ est indépendant de $B$ et $B$ de $C$ et si $A$ l'est de
    $B\cap C$, alors $C$ est indépendant de $A\cap B.$
    \end{enumerate}
\end{exo}

\begin{exo}
Montrer les propriétés suivantes quand elles sont vraies; dans le cas 
con\-trai\-re, construire un contre-exemple.
\begin{enumerate} 
\item
$\mathbb P(A|B)+\mathbb P(\overline{A}|B)=1.$ 
\item
$\mathbb P(A|B)+\mathbb P(A|\overline{B})=1.$ 
\item
$\mathbb P(A|B)+\mathbb P(\overline{A}|\overline{B})=1.$ 
\end{enumerate}
\end{exo}


\begin{exo}
Montrer que si $A$ et $B$ sont deux événements, $\mathbb P(A \cap B)-\mathbb P(A)\mathbb P(B)=-\mathbb P(\overline{A} \cap B)+\mathbb P(\overline{A})\mathbb P(B)=\mathbb P(\overline{A} \cap \overline{B})-\mathbb P(\overline{A})\mathbb P(\overline{B})$.
En déduire que si $A$ et $B$ sont indépendants, alors $\overline{A}$ et $B$ sont indépendants, et $\overline{A}$ et $\overline{B}$ sont indépendants.
\end{exo}

\subsection*{Exercices variés...}

\begin{exo} On lance un dé à six faces. Est-ce que les évènements \og obtenir un
résultat pair\fg{}  et \og obtenir un résultat divisible par 3\fg{} sont
indépendants?
\end{exo}


\begin{exo} La probabilité pour qu'un étudiant $A$ résolve un certain problème est de 1/2~; pour un autre  étudiant $B$ cette probabilité est de 2/3. Quelle est la probabilité que le problème soit résolu par au moins un des deux  étudiants, s'ils travaillent séparément ?
\end{exo}


\begin{exo}
On admet qu'un chasseur tue un renard d'un coup de fusil une fois
sur trois. Six chasseurs font une battue et tirent en même temps sur le renard (on admettra que les
tirs sont mutuellement indépendants). Calculer la probabilité que le renard soit tué.
\end{exo}


\begin{exo}
Une boîte contient $r$ boules rouges et $b$ boules blanches. On tire deux boules sans remise. 
\\
\textbf{a.} Si la première est rouge, quelle est la probabilité que la 2ème le soit ? 
\\
\textbf{b.} On doit parier sur le r\'esultat du premier tirage, qui a d\'eja eu lieu en cachette, et on conna\^\i t le r\'esultat du deuxi\`eme tirage (qui a donn\'e rouge) : calculer la probabilit\'e que la premi\`ere boule ait  \'et\'e rouge, en tenant compte de cette indication.
\end{exo}


\begin{exo}
Il existe deux routes entre Metz et Nancy, et deux routes entre Nancy et Épinal. Chaque route est bloquée par la neige avec probabilité $p$, indépendamment des autres routes. Quelle est la probabilité qu'on puisse aller de Metz à Épinal ?
\end{exo}


\begin{exo} 
Mes voisins ont deux enfants. Quelle est la probabilité que les
deux enfants soient des filles :
\begin{enumerate}
\item  si je n'ai pas d'autre information,
\item  sachant que j'ai rencontré l'aînée des deux enfants et que
c'est une fille,
\item  sachant que l'un des deux enfants est une fille.
\end{enumerate}
\end{exo}


\begin{exo}Soit $B_{1}$ et $B_{2}$ deux bo\^\i tes contenant respectivement $r_{1}$ boules rouges et $n_{1}$ boules noires, $r_{2}$ boules rouges et $n_{2}$ boules noires. On tire au hasard (c'est-\`a-dire avec  \'equiprobabilit\'e) l'une des 2 bo\^\i tes, et ensuite, dans la bo\^\i te choisie, on tire une boule, les choix  \`a  l'int\'erieur de la bo\^\i te  \'etant  \`a  nouveau  \'equiprobables. Sachant qu'on a tir\'e une boule rouge, quelle est la probabilit\'e de l'avoir tir\'ee de la bo\^\i te $B_{1}$ ?
\end{exo}


\begin{exo} \`A une question d'examen, quatre r\'eponses possibles sont propos\'ees au
candidat, et une seule est correcte. Si un \'etudiant a travaill\'e cette question, il est
s\^ur de r\'epondre correctement, sinon il choisit une des 4 r\'eponses au hasard. Le
programme de l'examen comporte 1000 questions et l'\'etudiant en a travaill\'e 600.

L'\'etudiant ayant r\'epondu correctement \`a la question pos\'ee, quelle est la probabilit\'e
pour qu'il ait travaillé cette question ?
\end{exo}


\begin{exo}
Dans un restaurant de 50 places, la probabilit\'e pour qu'une
personne ayant r\'eserv\'e une table, ne vienne pas est de 1/5. Un
jour, le patron a pris 52 r\'eservations. Quelle est la probabilit\'e
qu'il se trouve dans une situation embarrassante ?
\end{exo}

\begin{exo} Combien de fois faut-il lancer un d\'e \'equilibr\'e pour que la probabilit\'e d'obtenir au moins une fois un 6 soit sup\'erieure \`a 90 \% ? (les lancers successifs
sont suppos\'es ind\'ependants)
\end{exo}


\begin{exo} Une urne contient $N$ jetons num\'erot\'es de 1  \`a $N$. On tire au hasard $n$ jetons (avec $1\leq n\leq N$) de cette urne, sans remise. On appelle $X$ le plus petit num\'ero tir\'e. Quelle est la probabilit\'e que $X$ soit  \'egal  \`a $k$, en fonction de $k$, pour $k$ compris entre 1 et $N-n+1$ ?
\end{exo}

\begin{exo} Une personne sur 10 a une propension g\'en\'etique
\`a devenir alcoolique. Si c'est le cas, elle deviendra alcoolique avec probabilité 0,3; sinon,
elle deviendra alcoolique avec probabilité 0,05 (donn\'ees imaginaires !).

Sachant qu'une personne est alcoolique, quelle est la probabilité qu'elle n'ait pas de propension
g\'en\'etique \`a le devenir ?
\end{exo}

\begin{exo}
L'{\'e}tudiant X vient en TD deux fois sur trois, et quand il vient, il arrive une fois sur deux {\`a} l'heure, les autres fois il a au moins vingt minutes de retard. Il est 8h40, le TD commence {\`a} 8h30. Quelle chance a le professeur de voir encore arriver X?
\end{exo}


\begin{exo}Toutes les minutes un phare  \'emet un signal lumineux avec probabilit\'e $p$ ($0<p<1$). Un observateur lointain d\'etecte le signal, lorsque celui-ci est  \'emis, avec la probabilit\'e $\ell$. Calculer la probabilit\'e qu'\`a  l'instant $k$, l'observateur observe pour la premi\`ere fois le signal.
\end{exo}

\end{document}