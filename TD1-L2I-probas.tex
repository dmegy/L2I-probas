\documentclass[11pt,a4paper]{article}

\usepackage[utf8]{inputenc} % ncessaire sur PC
\usepackage[T1]{fontenc}

\usepackage{mathtools,amssymb,amsthm}
\usepackage{ifthen}
\usepackage{verbatim}
\newboolean{enonce} 
%\setboolean{enonce}{false}
\setboolean{enonce}{true}
\newboolean{indication} 
%\setboolean{indication}{true}
\setboolean{indication}{false}
%\usepackage{eepic}

\usepackage[dvips]{graphicx}

\usepackage[english,frenchb]{babel}
%\usepackage{eurosym}
\usepackage{fancyhdr}

\newcounter{encompt}

\theoremstyle{definition}
\newtheorem{exo}{Exercice}

\newcommand{\refex}[1]{\thechapter.\ref{#1}}
\pagestyle{empty}
\newcommand{\ind}{ 1\hspace{-.55ex}\mbox{l}}
\setlength{\topmargin}{-0.5in}
\setlength{\textheight}{9in}
\setlength{\oddsidemargin}{0in}
\setlength{\evensidemargin}{0in}
\setlength{\textwidth}{6.5in}

\newcommand{\Card}{\mathrm{Card \,}}
\def\N{\mathbb N}
\def\R{\mathbb R}
\def\E{\mathbb E}
\renewcommand{\P}{\mathbb P}
\def\Sn{\frak{S}_n}

\newcommand{\titre}[1]{\pagestyle{fancy}\lhead{Probabilit\'es \hfill L2 Informatique -- S4} 
\rhead{}\begin{center}
\textbf{#1}
\end{center}\bigskip}


\begin{document}

\titre{Feuille 1 : Introduction aux probabilités et à la combinatoire}


\begin{exo} Considérons un jeu de 5 cartes, numérotées de 1 à 5. 
\begin{enumerate}
\item
Première expérience aléatoire~: je pioche une carte et je relève son
numéro. Donner l'univers correspondant, et la partie correspondant à
l'événement \og le résultat obtenu est strictement plus grand que 3\fg.
\item
Deuxième expérience aléatoire~: je pioche une première carte, et,
sans la remettre, j'en pioche une deuxième~; je relève, dans l'ordre,
les deux numéros obtenus. Donner l'univers correspondant, et la partie correspondant à
l'événement \og la deuxième carte a un numéro plus grand que la
première\fg.
\item
Troisième expérience aléatoire~: je pioche deux cartes en même
temps, et je relève, sans ordre, les deux numéros obtenus. Donner l'univers correspondant, et la partie correspondant à
l'événement \og les deux numéros sont supérieurs ou égaux à 3\fg.
\item
Quatrième expérience aléatoire~: je pioche une première carte, et,
après l'avoir remise, j'en pioche une deuxième~; je relève, dans l'ordre,
les deux numéros obtenus. Donner l'univers correspondant, et la partie correspondant à
l'événement \og la deuxième carte a un numéro strictement plus grand que la
première\fg.
\end{enumerate}

 \emph{Bien remarquer les différences entre les trois
dernières expériences~: avec ou sans remise, ordonnée ou
non. Profitons-en pour rappeler la différence entre un  \emph{couple} et
une \emph{paire}~: un couple se note entre parenthèses, et ses
éléments sont ordonnés. Ainsi $(2,7)$ et $(7,2)$ sont deux couples
différents. En revanche, une paire est une partie à deux éléments
non ordonnée~: il n'y a qu'une seule paire composée des éléments 7 et
2, qu'on note entre accolades $\{2,7\}$ ou $\{7,2\}$.}
\end{exo}


\begin{exo}
Deux personnes A et B entrent dans une pièce où il y a 3 bancs de 2 places et s'assoient ``au hasard''. Quelle est la probabilité $p$ que A et B soient assises sur le même banc ? On peut par exemple :
\begin{enumerate}
\item numéroter les places de 1 \`a 6. A tire alors un numéro, le garde et B tire alors un numéro parmi les 5 restants,
\item num\'eroter les bans de 1 \`a 3. A tire alors un num\'ero, le remet puis B tire \`a son tour un des trois num\'eros.
\end{enumerate}

Montrer que les 2 fa\c cons de proc\'eder conduisent \`a 2 valeurs diff\'erentes de $p$.  

 \emph{Bilan : quand on rencontre l'expression ``choisir au hasard'', il est important d'expliciter clairement les hypothèses d'équiprobabilité sous-entendues.}
\end{exo}


\begin{exo}
\begin{enumerate}
\item On lance trois pièces de monnaie. Calculer les probabilités des
évènements suivants : $A = \{\text{il y a exactement 2 ``faces''} \}$ et 
$B = \{\text{il y au moins 2 ``faces''} \}$.
\item On lance deux dés non truqués, un rouge et
un bleu. Quelle est la probabilité que le résultat du dé rouge soit
pair et le résultat du dé bleu soit divisible par 3~?
\end{enumerate}
\end{exo}


\begin{exo}[(Paradoxe de Galton)] On lance trois fois de suite une pièce équilibrée. Que pensez- vous du raisonnement suivant ?
``A coup sûr, il y a deux lancers identiques. Par symétrie, le troisième a la même probabilité d'être un pile ou un face. Donc la probabilité d'avoir trois lancers identiques est 1/2.''
\end{exo}


\begin{exo}[(Probl{\`e}me de Galil{\'e}e)]
Le prince de Toscane demande un jour {\`a} Galil{\'e}e : ``Pourquoi lorsqu'on jette
 trois d{\'e}s obtient-on plus souvent la somme $10$ que la somme $9$, bien que
 ces deux sommes soient obtenues de six fa{\c c}ons diff{\'e}rentes ?''
  
Construire un mod{\`e}le probabiliste, traduire la question pos{\'e}e en terme de probabilit\'es et r\'epondre \`a la question.
Quelle erreur commet implicitement le prince de Toscane ?
\end{exo}


\begin{exo} 
On se propose dans cet exercice de répondre aux questions suivantes sur 
les dénombrements élémentaires.
\\
\textbf{a.}
{\sl Suites avec répétition, tirages avec remise}.
      \begin{enumerate}
      \item Combien peut-on écrire de mots de $n$ lettres avec un alphabet 
            de $r$ lettres ?
      \item Combien y a-t-il de façons de ranger $n$ boules numérotés dans $r$
            boites ?
      \item Soit une urne comportant $n$ boules numérotées, 
            on tire $r$ fois une boule dans l'urne, on note son numéro (à la $i$--ème 
            position d'une liste lors du $i$--ème tirage), puis on la remet dans
            l'urne. Combien y a-t-il de listes différentes ?
      \end{enumerate}
\textbf{b.} 
{\sl Suites sans répétition, tirages sans remise}.
      \begin{enumerate}
      \item Même contexte qu'en a-(1) mais cette fois, on comptabilise uniquement
            les mots comportant des lettres toutes distinctes.
      \item Même contexte qu'en a-(2) mais on ne peut mettre qu'une boule par boite
            au maximum.
      \item Même contexte qu'en a-(3) sauf que l'on ne remet pas la boule
            dans l'urne à chaque tirage (tirage sans remise).
      \end{enumerate}
\textbf{c.} 
{\sl Combinaisons}.
      \begin{enumerate}
      \item Combien y a-t-il de façons de ranger $n$ boules numérotés dans $2$
            boites de telle façon qu'il y en ait $k$ dans l'une et $n-k$ dans l'autre ?
      \item Avec un alphabet de $2$ lettres, combien de mots de $n$ lettres  
            comportant $k$ fois l'une et $n-k$ fois l'autre, peut-on former ?
      \end{enumerate}
\end{exo}



\begin{exo}
Combien de mots peut-on former en m\'elangeant les lettres du mot PARIS ? M\^eme question avec NANCY, TOULOUSE et NARBONNE.
\end{exo}


\begin{exo}
On tire d'un seul coup trois cartes d'un jeu de 32. Calculer la probabilit\'e des \'ev\`enements suivants : 
\begin{enumerate}
\item la main est compos\'ee de trois as, 
\item la main est compos\'ee de 2 as et une dame, 
\item la main est compos\'ee d'au moins un valet.
\end{enumerate}
\end{exo}


\begin{exo}
Dans une assembl\'ee de $n$ personnes, on note la date d'anniversaire de chacun.
\begin{enumerate}
\item Proposer un ensemble $\Omega$ pour coder l'ensemble des r\'esultats possibles, et donner son cardinal.
\item Calculer le cardinal des ensembles suivants:
\begin{enumerate}
\item tous les participants ont la m\^eme date d'anniversaire,
\item Pierre et Paul ont la m\^eme date d'anniversaire,
\item toutes les personnes ont des dates d'anniversaires diff\'erentes,
\item il y a au moins deux personnes qui ont la m\^eme date d'anniversaire.
\end{enumerate}
\item On note $p_n$ la probabilit\'e pour que deux personnes au moins de cette assembl\'ee aient 
leur anniversaire le m\^eme jour. Calculer une valeur approch\'ee de cette probabilit\'e pour $n=23$, $n=41$ et $n=57$.
\end{enumerate}
\end{exo}

\end{document}