\documentclass[11pt]{article}
\usepackage[utf8]{inputenc}
\usepackage{fourier}
\usepackage[margin=2cm]{geometry}
\usepackage[french]{babel}

\usepackage{mathtools,amssymb,amsthm,comment,xcolor}
\renewcommand{\P}{\mathbb P}
\newcommand{\E}{\mathbb E}
\newcommand{\R}{\mathbb R}
\newcommand{\V}{\mathbb V}

\theoremstyle{definition}
\newtheorem{exo}{Exercice}

% pour afficher les solutions : 
\newenvironment{solution}{\begin{quote}\color{teal}}{\end{quote}}
% pour masquer les solutions:
%\excludecomment{solution}


\begin{document}
\noindent Licence d'informatique \hfill Faculté des sciences de Nancy\\
\noindent\rule{\linewidth}{1pt}
\begin{center}
Probabilités et statistiques\\
Épreuve du 18 mai 2022\\
Durée 1h30 --- Calculatrices et documents interdits\\
Attention : toutes les réponses doivent être justifiées.
\end{center}
\noindent\rule{\linewidth}{1pt}




\begin{exo}
Les diagonales d'un polygone régulier sont les segments qui relient deux sommets non consécutifs. Par exemple, un carré possède deux diagonales, et un pentagone régulier possède cinq diagonales.

\begin{enumerate}
\item Combien de diagonales un polygone régulier à 6 côtés possède-t-il ?
\item On fixe $N\geq 4$. Combien de diagonales un polygone régulier à $N$ côtés possède-t-il ? (Vérifiez votre résultat avec les calculs précédents!)
\end{enumerate}
\begin{solution}
\begin{enumerate}
\item On compte les diagonales et on en trouve neuf.
\item Il y a $\binom{N}{2}$ façons de choisir deux sommets du polygone et de les relier. Parmi ces segments, il y en a $N$ qui sont les $N$ côtés du polygone. Le polygone possède donc $\binom{N}{2}-N = \frac{N(N-1)}{2}-N = \frac{N(N-3)}{2}$ diagonales.\end{enumerate}
\end{solution}
\end{exo}


\begin{exo}
On jette deux dés à six faces équilibrés.
\begin{enumerate}
\item Quelle est la probabilité d'avoir obtenu au moins un cinq, sachant que la somme des deux dés est égale à dix ?
\item Quelle est la probabilité d'avoir obtenu au moins un cinq, sachant que les résultats des deux dés sont différents ?
\end{enumerate}
\begin{solution}
\begin{enumerate}
\item Notons $A$ l'évènement \og avoir obtenu au moins un cinq\fg{} et $B$ l'évènement \og la somme des deux dés vaut $10$\fg. Alors $\P(A\cap B) = \frac{1}{36}$ et $\P(B)) = \frac{3}{36}$. Donc $P(A|B) = \frac{1}{3}$.
\item Notons $C$ l'évènement \og avoir obtenu deux résultats différents\fg. On a $\P(A\cap C)=\frac{10}{36}$  et $\P(C) = \frac{30}{36}$, donc $\P(A|C)=\frac{10}{30} = \frac{1}{3}$.
\end{enumerate}
\end{solution}
\end{exo}



\begin{exo}
On considère une pièce de monnaie truquée, qui a une probabilité $1/3$ de retomber sur pile, et une probabilité $2/3$ de retomber sur face. On joue au jeu suivant. Le joueur paye $5$ euros pour avoir le droit de jouer. Dans ce cas, il lance trois fois la pièce et gagne $6$ euros à chaque fois qu'il obtient pile.
On appellera $X$ la variable aléatoire représentant le nombre de pile.
\begin{enumerate}
\item Quelles valeurs peut prendre $X$ ? Donner la loi de $X$.
\item Calculer l'espérance de $X$.
\item Est-il rentable de jouer ?
% espérance=1
\end{enumerate}
\begin{solution}
\begin{enumerate}
\item $X$ peut prendre les valeurs $0$, $1$, $2$ et $3$. La loi de $X$ est:
\[
\P(X=0) = \left(\frac{2}{3}\right)^3=\frac{8}{27};\quad
\P(X=1) = 3\times \frac{1}{3}\times  \left(\frac{2}{3}\right)^2 = \frac{4}{9};\quad
\P(X=2) = 3\times \frac{2}{3}\times  \left(\frac{1}{3}\right)^2 = \frac{2}{9};\quad
\P(X=3) = \left(\frac{1}{3}\right)^3=\frac{1}{27}.
\]
(On vérifie bien sûr que la somme de ces probabilités vaut bien $1$ : $\frac{8}{27}+\frac{12}{27} +\frac{6}{27}+\frac{1}{27}=1$.)
\item L'espérance de X est $\E(X) =0\times  \frac{8}{27}+1\times \frac{12}{27} +2\times \frac{6}{27}+3\times \frac{1}{27} = \frac{27}{27}=1$. (On pouvait aussi reconnaître une loi binomiale de paramètres $n=3$ et $p=\frac{1}{3}$, l'espérance est donc $np=1$.)
\item Le gain total du joueur est $Y=-5+6X$ (on paye $5$ euros pour jouer, et on gagne $X$ fois six euros). On en déduit que l'espérance de gain est 
$\E(Y) = \E(-5+6X) = -5+6\E(X) = 1$.
Il est donc rentable de jouer à ce jeu.
\end{enumerate}
\end{solution}
\end{exo}

 \begin{exo} 
 Une compagnie d'assurances a classé ses assurés en trois catégories suivant le risque : faible, moyen, et haut. Les statistiques montrent que la probabilité d'avoir un accident sur une période d'un an sont de $5\%$, $15\%$ et $30\%$ dans ces trois catégories. D'autre part on observe que $20\%$ des assurés sont dans la catégorie à bas risque, $50\%$ à risque moyen, et $30\%$ à haut risque.
 \begin{enumerate}
 \item Quelle est la probabilité d'avoir un accident (sur la période d'un an) ? 
 \item Un certain assuré n'a pas d'accident sur la période d'un an. Quelle est la probabilité qu'il appartienne à la catégorie de risque faible ?
 \end{enumerate}
 
 \begin{solution}
 Notons $F$, $M$ et $H$ les évènements correspondants aux catégories de risque faible, moyen et haut. Notons également $A$ l'évènement correspondant à un accident.
 L'énoncé donne les informations suivantes :
 \[
 \P(A|F) = \frac{5}{100}=\frac{1}{20}, \:
 \P(A|M) = \frac{15}{100}=\frac{3}{20}, \:
 \P(A|H)=\frac{30}{100}=\frac{3}{10},
 \]
 \[
 \P(F)=\frac{20}{100}=\frac{1}{5},\:
 \P(M)=\frac{50}{100} =\frac{1}{2},\:
 \P(H)=\frac{30}{100}=\frac{3}{10}.
 \]
 Le système des différents risques est un système complet d'évènements (les assuré sont dans une des trois catégories et une seule).
\begin{enumerate}
\item On écrit
\begin{align*}
\P(A)
= \P(A|F)\P(F) + \P(A|M)\P(M)+\P(A|H)\P(H)
&= \frac{1}{20}\times\frac{1}{5} + \frac{3}{20}\times\frac{1}{2}+\frac{3}{10}\times\frac{3}{10}\\
&= \frac{1}{100}+\frac{3}{40}+\frac{9}{100}\\
&= \frac{2+15+18}{200} = \frac{35}{200}=\boxed{\frac{7}{40}} \quad (=0,175)
\end{align*}
\item On veut calculer $\P(F|\bar A)$. On utilise la méthode habituelle, sachant que $\P(\bar A)$ se déduit de la question précédente :
\begin{align*}
\P(F|\bar A) = \frac{\P(F\cap \bar A)}{\P(\bar A)} 
= \frac{\P(\bar A | F)\P(F)}{\P(\bar A)}
= \frac{\P(\bar A | F)\P(F)}{1-\P(A)}
=\frac{\frac{19}{20}\times\frac{1}{5}}{\frac{33}{40}}
= \frac{19}{20}\times\frac{1}{5}\times\frac{40}{33}=\boxed{\frac{38}{165}} \quad (\sim 0,23)
\end{align*}
\end{enumerate}
\end{solution}
 \end{exo}

\begin{exo}
     Trois chasseurs tirent simultan\'ement sur 3 canards. On suppose que chaque chasseur , ind\'epen-damment des autres, choisit un canard au hasard et le tue.
     On appelle $N$ la variable al\'eatoire \'egale au nombre de canards tu\'es. 
     \begin{enumerate}
     \item Calculer la loi de $N$.
     \item Calculer l'espérance $\mathbb E(N)$.
     \end{enumerate}
     
     \begin{solution}
\begin{enumerate}
\item On a $\P(N=1)=3\times \frac{1}{27}=\frac{1}{9}$, $\P(N=3)=\frac{6}{27}$ ((il y a six permutations possibles pour les trois chasseurs), et donc, puisque la somme des probabilités doit valoir un, on a $\P(N=2)=\frac{18}{27}$. (Alternativement on peut compter les cas et ensuite vérifier que la somme des probabilités est bien égale à $1$.)
\item \textbf{Première méthode} : calcul de l'espérance avec la formule du cours. On a 
\[ \E(N) = \frac{1}{9}+2\times \frac{18}{27}+3\times\frac{6}{27} = \frac{3+36+18}{27} = \frac{57}{27}=\boxed{\frac{19}{9}}
\]
\textbf{Deuxième méthode}, sans utiliser la première question:\\
Notons $C_1$ la variables aléatoire qui vaut $1$ si le premier canard est touché et $0$ sinon, et de même pour $C_2$ et $C_3$. La probabilité que  le premier canard survive vaut $\P(C_1=0)=(2/3)^3$, donc on a $\E(C_1)=1-(2/3)^3=\frac{19}{27}$. Les variables $C_2$ et $C_3$ ont la même loi et la même espérance puisque la situation est la même pour tous les canards. D'autre part, on a  $N=C_1+C_2+C_3$ et donc par linéarité de l'espérance, 
$\E(N) = \E(C_1)+\E(C_2)+\E(C_3) = 3\E(C_1) =3\times \frac{19}{27} =\boxed{\frac{19}{9}}$
\end{enumerate}
\end{solution}
\end{exo}


\begin{exo}

On définit une fonction $g : \mathbb R\to \mathbb R$ de la façon suivante : si $t<-1$ ou $t>1$, alors $g(t)=0$. Par contre, si $t\in [-1,1]$, alors $g(t)=(1-t^2)^2$.

\begin{enumerate}
\item Que vaut $\displaystyle{ \int_{-\infty}^{+\infty} g(t)d\! t}$ ?
\item Soit $C$ la constante telle que $\int_{-\infty}^{+\infty} Cg(t)dt=1$. Dans la suite, on définit la fonction $f$ par $f(t)=Cg(t)
$ pour tout $t$. Montrer que $f$ est une densité de variable aléatoire réelle continue, que l'on note $X$.
\item Calculer l'espérance de $X$.
\item Calculer la variance de $X$.
\end{enumerate}

\begin{solution}
\begin{enumerate}
\item On a
\[ \int_{-\infty}^{+\infty} g(t)d\! t= \int_{-\infty}^{-1} g(t)d\! t+ \int_{-1}^{1} g(t)d\! t+ \int_{1}^{+\infty} g(t)d\! t  = 
0+\int_{-1}^{1} g(t)d\! t+0
=\int_{-1}^{1} (1-t^2)^2d\! t
\]
\[
=\int_{-1}^1 (t^4-2t^2+1)d\!t = \left[ \frac{t^5}{5} -\frac{2t^3}{3} +t \right]_{-1}^1=\boxed{\frac{16}{15}}\]
\item D'après la question précédente,  on a $C=\frac{15}{16}$. On vérifie les trois points du \og cahier des charges\fg{} des densités de probabilités : la fonction $f$ est positive, continue par morceaux, et son intégrale sur $\R$ vaut $1$.
\item L'espérance de la variable aléatoire $X$ vaut d'après le cours 
\[ \E(X) = \int_{-\infty}^{+\infty} tf(t)d\!t
=\int_{-1}^{1} (t^5-2t^3+t)d\!t=0,
\]
puisque  l'on intègre une fonction impaire entre $-1$ et $1$. Le résultat est cohérent avec le fait que la densité de probabilité est paire, c'est-à-dire symétrique par rapport à l'abscisse $x=0$.
\item Par définition, la variance de $X$ vaut 
\[ \V(X) = \E(X^2)-\E(X)^2 = \E(X^2) =  \int_{-\infty}^{+\infty} t^2f(t)d\!t 
= \int_{-1}^{1}(t^6-2t^4+t^2)d\!t
\]
\[
= \left[ \frac{t^7}{7} - \frac{2t^5}{5} + \frac{t^3}{3}\right]_{-1}^{1}
=2\left( \frac{1}{7}-\frac{2}{5}+\frac{1}{3}\right)
=2\left(\frac{15-42+35}{105}\right) = \boxed{\frac{16}{105}}
\]
\end{enumerate}
\end{solution}
\end{exo}

\end{document}