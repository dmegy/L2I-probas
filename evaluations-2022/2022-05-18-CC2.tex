\documentclass[11pt]{article}
\usepackage[utf8]{inputenc}
\usepackage{fourier}
\usepackage[margin=2cm]{geometry}
\usepackage[french]{babel}

\usepackage{mathtools,amssymb,amsthm,comment,xcolor}
\renewcommand{\P}{\mathbb P}

\theoremstyle{definition}
\newtheorem{exo}{Exercice}

% pour afficher les solutions : 
\newenvironment{solution}{\begin{quote}\color{teal}}{\end{quote}}
% pour masquer les solutions:
%\excludecomment{solution}


\begin{document}
\noindent Licence d'informatique \hfill Faculté des sciences de Nancy\\
\noindent\rule{\linewidth}{1pt}
\begin{center}
Probabilités et statistiques\\
Épreuve du 18 mai 2022\\
Durée 1h30 --- Calculatrices et documents interdits\\
Attention : toutes les réponses doivent être justifiées.
\end{center}
\noindent\rule{\linewidth}{1pt}




\begin{exo}
Les diagonales d'un polygone régulier sont les segments qui relient deux sommets non consécutifs. Par exemple, un carré possède deux diagonales, et un pentagone régulier possède cinq diagonales.

\begin{enumerate}
\item Combien de diagonales un polygone régulier à 6 côtés possède-t-il ?
\item On fixe $N\geq 4$. Combien de diagonales un polygone régulier à $N$ côtés possède-t-il ? (Vérifiez votre résultat avec les calculs précédents!)
\end{enumerate}
\end{exo}


\begin{exo}
On jette deux dés à six faces équilibrés.
\begin{enumerate}
\item Quelle est la probabilité d'avoir obtenu au moins un cinq, sachant que la somme des deux dés est égale à dix ?
\item Quelle est la probabilité d'avoir obtenu au moins un cinq, sachant que les résultats des deux dés sont différents ?
\end{enumerate}
\end{exo}



\begin{exo}
On considère une pièce de monnaie truquée, qui a une probabilité $1/3$ de retomber sur pile, et une probabilité $2/3$ de retomber sur face. On joue au jeu suivant. Le joueur paye $5$ euros pour avoir le droit de jouer. Dans ce cas, il lance trois fois la pièce et gagne $6$ euros à chaque fois qu'il obtient pile.
On appellera $X$ la variable aléatoire représentant le nombre de pile.
\begin{enumerate}
\item Quelles valeurs peut prendre $X$ ? Donner la loi de $X$.
\item Calculer l'espérance de $X$.
\item Est-il rentable de jouer ?
% espérance=1
\end{enumerate}
\end{exo}

 \begin{exo} 
 Une compagnie d'assurances a classé ses assurés en trois catégories suivant le risque : faible, moyen, et haut. Les statistiques montrent que la probabilité d'avoir un accident sur une période d'un an sont de $5\%$, $15\%$ et $30\%$ dans ces trois catégories. D'autre part on observe que $20\%$ des assurés sont dans la catégorie à bas risque, $50\%$ à risque moyen, et $30\%$ à haut risque.
 \begin{enumerate}
 \item Quelle est la probabilité d'avoir un accident (sur la période d'un an) ? 
 \item Un certain assuré n'a pas d'accident sur la période d'un an. Quelle est la probabilité qu'il appartienne à la catégorie de risque faible ?
 \end{enumerate}
 \end{exo}

\begin{exo}
     Trois chasseurs tirent simultan\'ement sur 3 canards. On suppose que chaque chasseur , ind\'epen-damment des autres, choisit un canard au hasard et le tue.
     On appelle $N$ la variable al\'eatoire \'egale au nombre de canards tu\'es. 
     \begin{enumerate}
     \item Calculer la loi de $N$.
     \item Calculer l'espérance $\mathbb E(N)$.
     \end{enumerate}
\end{exo}


\begin{exo}

On définit une fonction $g : \mathbb R\to \mathbb R$ de la façon suivante : si $t<-1$ ou $t>1$, alors $g(t)=0$. Par contre, si $t\in [-1,1]$, alors $g(t)=(1-t^2)^2$.

\begin{enumerate}
\item Que vaut $\displaystyle{ \int_{-\infty}^{+\infty} g(t)d\! t}$ ?
\item Soit $C$ la constante telle que $\int_{-\infty}^{+\infty} Cg(t)dt=1$. Dans la suite, on définit la fonction $f$ par $f(t)=Cg(t)
$ pour tout $t$. Montrer que $f$ est une densité de variable aléatoire réelle continue, que l'on note $X$.
\item Calculer l'espérance de $X$.
\item Calculer la variance de $X$.
\end{enumerate}
\end{exo}

\end{document}